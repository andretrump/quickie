\chapter{Programming Principles}

\section{SOLID}

\section{GRASP}

\section{DRY}
DRY steht für \enquote{Don't Repeat Yourself} und ist der Name eines Programmierprinzips, welches besagt, dass redundanter Code vermieden werden sollte. Grund dafür ist, dass dadurch Wissen zentral festgehalten wird. Sollte das festgehaltene Wissen verändert werden müssen, so muss dieses folglich auch nur an einer Stelle geändert werden. Im Folgenden sollen Code-Beispiele diskutiert werden, welche Positiv- und Negativbeispiele für die Anwendung des DRY-Prinzips darstellen.

Ein Beispiel für das Einhalten des DRY-Prinzips ist in der Verifizierung der Parameter eines Callbacks zu sehen. Hier wird zentral in jeder Callback-Klasse durch die Methode \code{getRequiredParameters} definiert, welche Parameter benötigt werden. Das CLI Plugin nutzt diese Informationen, um die übergebenen Parameter in der Methode \code{ensureRequiredParameters} auf Vollständigkeit für den entsprechenden Callback zu prüfen sowie um die Benutzerhilfe in \code{constructOptions} zu generieren.

Ein Negativbeispiel für das Nichteinhalten des DRY-Prinzips ist in der Implementierung der zu persistierenden Entitäten zu sehen. Diese werden einmal zentral in der Domäne definiert, während es dann jedoch in der Adapterschicht noch die Interfaces \code{PersistentProfile} und \code{PersistentRecipe} gibt, welche durch Getter- und Setter-Methoden die notwendigen und zu persistierenden Eigenschaften der Entitäten definieren. Diese werden abschließend im \code{persistence} Plugin implementiert. Dadurch wird die Struktur einer Entität einmal in der Domäne und einmal in der Persistenz definiert, es gibt eine Dopplung der Informationen, welche Attribute zu einer Entität gehören. Diese Verletzung des DRY-Prinzips wurde begangen, um die Persistenz-Darstellung von der Arbeitsdarstellung zu entkoppeln, wodurch die Persistenz deutlich flexibler gestaltet werden kann.

Ein weiteres simpleres Negativbeispiel findet sich in Methoden des \code{JPAProfileRepository} und \code{JPARecipeRepository}. In diesen wird, falls ein \code{PersistentReicipe} oder \code{PersistentProfile} übergeben wird, immer zunächst geprüft, ob dieses eine Instanz eines \code{JPARecipe} beziehungsweise \code{JPAProfile} ist. Diese Überprüfung hätte in eine eigene Methode ausgelagert werden, um Flüchtigkeitsfehler durch die Redundanz des Codes zu verhindern und die Wartung zu erleichtern.

Ein Beispiel für eine solche ausgelagerte Methode findet sich in der Klasse \code{CommandLineUI} mit der Methode \code{printHelp}. Diese Methode definiert zentral, wie die Hilfe ausgegeben werden soll und erfüllt somit das DRY-Prinzip.