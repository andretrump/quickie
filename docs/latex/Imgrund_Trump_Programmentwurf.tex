% ------------------------------------------------------------
% LaTeX Template für die DHBW zum Schnellstart!
% Original: https://github.wdf.sap.corp/vtgermany/LaTeX-Template-DHBW
% ------------------------------------------------------------
% ---- Präambel mit Angaben zum Dokument
\input{Inhalt/00_Latex/praeambel}

% ---- Elektronische Version oder Gedruckte Version?
% ---- Unterschied: Die elektronische Version enthält keinen Platzhalter für die Unterschrift
\usepackage{ifthen}
\newboolean{e-Abgabe}
\setboolean{e-Abgabe}{false}    % false=gedruckte Fassung

% ---- Persönlichen Daten:
\newcommand{\titelheader}{Programmentwurf}
\newcommand{\bearbeitende}{Andr\'{e} Trump und Erik Imgrund}

\input{Inhalt/00_Latex/kopfundFusszeile}

% ---- Hilfreiches
\newcommand{\zB}{z.\,B. }   % "z.B." mit kleinem Leeraum dazwischen (ohne wäre nicht korrekt)
\newcommand{\dash}{d.\,h. }

\newcommand{\code}[1]{\texttt{#1}} % Ist einfacher zu schreiben als ständig \texttt und erlaubt
                                   % Änderungen im Nachhinein, wenn man z.B. Inline-Code anders stylen möchte.

% ---- Silbentrennung (falls LaTeX defaults falsch / nicht gewünscht sind)
\hyphenation{HANA}         % anstatt HA-NA
\hyphenation{Graph-Script} % anstatt GraphS-cript

% ---- Beginn des Dokuments
\begin{document}
\setlength{\parindent}{0pt}              % Keine Paragraphen Einrückung.
                                         % Dafür haben wir den Abstand zwischen den Paragraphen.
\setcounter{secnumdepth}{2}              % Nummerierungstiefe fürs Inhaltsverzeichnis
\setcounter{tocdepth}{1}                 % Tiefe des Inhaltsverzeichnisses. Ggf. so anpassen,
                                         % dass das Verzeichnis auf eine Seite passt.
\sffamily                                % Serifenlose Schrift verwenden.

% ---- Vorspann
% ------ Titelseite
\singlespacing
\thispagestyle{empty}
\begin{titlepage}
\enlargethispage{4cm}

\begin{figure}           % Logo vom Ausbildungsbetrieb und der DHBW
	% \vspace*{-5mm} % Sollte dein Titel zu lang werden, kannst du mit diesem "Hack" 
	%                  den Inhalt der Seite nach oben schieben.
	\begin{minipage}{0.49\textwidth}
		\flushleft
		%\includegraphics[width=0.9\textwidth]{Bilder/Logos/Logo.pdf} 
	\end{minipage}
	\hfill
	\begin{minipage}{0.49\textwidth}
		\flushright
		\includegraphics[height=2.5cm]{Bilder/Logos/Logo_DHBW.pdf} 
	\end{minipage}
\end{figure} 
\vspace*{0.1cm}

\begin{center}
	\huge{\textbf{Software-Engineering II}}\\[1.5cm]
	\Large{\textbf{Programmentwurf}}\\
	\Large{\textbf{TINF19B1}}\\
	\Large{\textbf{5.$+$6. Semester (2021/2022)}}\\[1cm]
	\Large{\textbf{Thema:}}\\
	\Large{\textbf{Matching von Kochrezepten}}\\[2cm]
\end{center}

\begin{center}
	\normalsize{Dozent:}\\
	\large{Daniel Lindner}
\end{center}

\begin{center}
	\normalsize{Bearbeitende:}\\
	\Large{\bearbeitende}
\end{center}
\end{titlepage}
  % Titelseite
\newcounter{savepage}
\pagenumbering{Roman}                    % Römische Seitenzahlen
\onehalfspacing

% ------ Inhaltsverzeichnis
\singlespacing
\tableofcontents

% ------ Verzeichnisse
\renewcommand*{\chapterpagestyle}{plain}
\pagestyle{plain}
\chapter*{Abkürzungsverzeichnis}
\addcontentsline{toc}{chapter}{Abkürzungsverzeichnis} % Hinzufügen zum Inhaltsverzeichnis 

\begin{acronym}[WYSISWG] % längstes Kürzel wird verw. für den Abstand zw. Kürzel u. Text

	% Alphabetisch selbst sortieren - nicht verwendete Kürzel rausnehmen!
	\acro{JPA}{Java Persistence API}

\end{acronym}

\listoffigures                          % Erzeugen des Abbildungsverzeichnisses 
\renewcommand{\lstlistlistingname}{Quellcodeverzeichnis}
\lstlistoflistings                      % Erzeugen des Listenverzeichnisses
\setcounter{savepage}{\value{page}}


% ---- Inhalt der Arbeit
\cleardoublepage
\pagenumbering{arabic}                  % Arabische Seitenzahlen für den Hauptteil
\setlength{\parskip}{0.5\baselineskip}  % Abstand zwischen Absätzen
\rmfamily
\renewcommand*{\chapterpagestyle}{scrheadings}
\pagestyle{scrheadings}
\onehalfspacing
\chapter{Beschreibung des Programms}
Im Internet ist eine Vielzahl an Rezepten zu finden, welche eine große Unterstützung beim Kochen und eine Erweiterung des eigenen kulinarischen Horizonts bieten können. Besonders hochwertig und vergleichsweise einfach in der Zubereitung sind die Rezepte auf der Webseite \url{https://www.hensslers-schnelle-nummer.de/}. Problematisch ist jedoch trotzdem, dass die Rezepte meist sehr verschiedene Zutaten benötigen und damit für jedes Rezept einzeln eingekauft werden müsste. Eine Planung mehrerer Rezepte ist für eine ganze Woche erweist sich allerdings ebenfalls als aufwendig, denn es müssen Rezepte mit übereinstimmenden Zutaten gefunden werden.

Die im Rahmen dieses Programmentwurfs entwickelte Software erfüllt den Nutzen, dass Rezepte mit möglichst großer Übereinstimmung gefunden und betrachtet werden können, was die Planung des Wocheneinkaufes erheblich erleichtert.

\section{Funktionalität}
Die Rezepte werden vollautomatisch aus der Webseite extrahiert und lokal persistiert. Dabei werden in der Datenbasis bereits vorhandene Rezepte übersprungen, um inkrementelle Aktualisierungen zu ermöglichen. Über eine Kommandozeilenanwendung können die Rezepte einzeln angezeigt, durchsucht und gefiltert werden.

Die Hauptfunktionalität besteht darin, zu jedem Rezept andere Rezepte mit möglichst vielen übereinstimmenden Zutaten zu finden bzw. beliebige Rezepte mit möglichst vielen übereinstimmenden Zutaten zu finden. 

Das Finden der übereinstimmenden Rezepte wird als Matching bezeichnet. Bei diesem Vorgang soll auch berücksichtigt werden, dass bestimmte Zutaten für einen Einkauf und damit auch für das Matching irrelevant sein können, wie beispielsweise Salz oder Pfeffer, da diese ohnehin in fast jedem Rezept vorkommen und daher in der Regel auch vorrätig sind. Daher soll es ermöglicht werden, im Programm zu hinterlegen, welche Zutaten vorrätig sind und daher beim Matching nicht berücksichtigt werden sollen. Da jeder Mensch, was Lebensmittel betrifft, einen individuellen Geschmack besitzt, soll auch dieser beim Matching der Rezepte berücksichtigt werden. Daher soll es für den Benutzer möglich sein, zu jedem Lebensmittel eine Meinung in verschiedenen Stufen von Brechreiz erregender Abneigung bis hin zum Ausbruch größter Glücksgefühle beim Konsum des jeweiligen Lebensmittels zu speichern.

Um eine Verwendung der Software durch verschiedene Nutzer zu ermöglichen, sollen die vorrätigen Lebensmittel sowie die Meinungen des Nutzers zu den verschiedenen Zutaten in Benutzerprofilen organisiert werden.

Als Benutzeroberfläche für die Software ist eine \ac{CLI} vorgesehen.

\section{Technologien}
Für die Umsetzung wird Java als Programmiersprache und Laufzeitumgebung verwendet. Für die Paketverwaltung wird Maven, für das Parsen des HTML Jsoup und zur Kommunikation mit der Datenbank werden JPA und Hibernate verwendet. Als Datenbankmanagementsystem wird H2 verwendet. Das Projekt befindet sich durch Git bzw. GitHub unter Versionsverwaltung. Der Source Code ist im Repository \href{https://youtu.be/dQw4w9WgXcQ}{https://github.com/anditru/quickie} öffentlich einsehbar.

\chapter{Domain Driven Design}

\section{Analyse der Ubiquitous Language}

\paragraph{Sprache} Als Sprache der Ubiquitous Language standen Deutsch und Englisch zur Auswahl. Da die Domänenexperten die Domäne ebenso gut in englischer wie in deutscher Sprache beschreiben können, Englisch jedoch von wesentlich mehr Personen und Entwicklern verstanden wird als Deutsch, wurde Englisch als Sprache für die Ubiquitous Language gewählt.

\paragraph{Wesentliche Begriffe} Die Domäne der Applikation betrifft im Wesentlichen die Themen \enquote{Kochrezepte} und \enquote{Lebensmittel}. Für die Begriffe der Domäne werden die bei den Domänenexperten im Englischen üblichen Begriffe verwendet.

Ein zentrales Konzept der Domäne ist das Kochrezept. Ein Kochrezept wird in der Ubiquitous Language mit dem Begriff \enquote{Recipe} bezeichnet. Ein Recipe besitzt typischerweise eine Liste von Zutaten, welche in der Domäne als \enquote{Ingredient} bezeichnet werden. Eine Ingredient bezeichnet ein Nahrungsmittel, von dem für ein Rezept eine bestimmte Menge verwendet wird, welche in einer bestimmten Einheit gemessen wird. Dieses Nahrungsmittel wird als \enquote{Food} bezeichnet, die verwendete Menge als \enquote{Quantity} und die entsprechende Einheit als \enquote{Unit}.

Das zweite zentrale Konzept der Domäne ist das Benutzerprofil, welches in der Ubiquitous Language als \enquote{Profile} bezeichnet wird. Zu einem Profile gehört, wie in der Einleitung erwähnt, eine Menge vorrätiger Foods. Neben den vorrätigen Foods gehören zu einem Profile auch die Meinungen des Nutzers zu bestimmten Lebensmitteln, welche als \enquote{Opinions} bezeichnet werden. Diese Opinion des Nutzers kann sieben Stufen annehmen: \enquote{Foodgasm}, \enquote{Love}, \enquote{Like}, \enquote{Indifferent}, \enquote{Dislike}, \enquote{Hate} und \enquote{Dealbeaker}, hier in zunehmender Reihenfolge der Abneigung des Nutzers gegenüber dem betreffenden Lebensmittel.

\section{Entities \& Value Objects}

\begin{figure}[ht!]
    \includegraphics[width=0.98\columnwidth]{../diagrams/domain_uml.pdf}
    \caption{Klassendiagramm der Domäne}
    \label{fig:class-diag-domain}
\end{figure}

\autoref{fig:class-diag-domain} zeigt das Klassendiagramm der Domäne. Im Package \href{https://github.com/anditru/quickie/tree/bb41442c7f1ffbfcd3117cd86a40f7932e543a33/3-quickie-domain/src/main/java/org/pinkcrazyunicorn/quickie/domain/recipe}{\code{recipe}} existiert eine Entity: das Recipe. Jedes Recipe besitzt durch eine UUID eine eigene Identität. Außerdem können Rezepte in seltenen Fällen auch nachträglich abgewandelt werden und besitzen daher grundsätzlich veränderliche Werte. Bei der Ingredient hingegen handelt es sich um ein Value Object, da diese keine eigene Identität besitzt. Existieren zwei Ingredients mit dem gleichen Food und der gleichen Quantity, so sind diese gleich. Außerdem ist ihr Zustand unveränderlich, so wird etwa bei einer Änderung in einem Rezept die betreffende Ingredient gelöscht und eine neue erstellt. Ebenso verhält es sich mit der Quantity: Sind Amount und Unit einer Quantity gleich, so sind auch die Quantities gleich und auch ihr Zustand ist unveränderlich. Zuletzt enthält das Package noch die Unit. Auch hier handelt es sich um ein Value Object, da auch die Gleichheit von Units lediglich von ihren Werten, also von ihren Namen abhängt. Ebenso besitzt eine Unit keinen eigenen Lebenszyklus und ihr Zustand ist unveränderlich.

Auch im Package \href{https://github.com/anditru/quickie/tree/bb41442c7f1ffbfcd3117cd86a40f7932e543a33/3-quickie-domain/src/main/java/org/pinkcrazyunicorn/quickie/domain/recipe}{\code{profile}} existiert eine Entity: das Profile. Jedes Profile besitzt durch seinen Namen eine eigene Identität, da auch zwei Profiles mit gleichem vorhandenen Food und gleichen Opinions zu verschiedenen Nutzern gehören können und daher unterscheidbar sein müssen. Die auch die Unterklassen der abstrakten Klasse Opinion, welche die möglichen Opinions eines Nutzers über ein Food abbilden, repräsentieren Value Objects, da sie weder eine eigene Identität noch veränderliche Werte besitzen. Da sie außerdem keine Attribute besitzen, wird ihre Gleichheit nur über ihren Typ bestimmt.

Zwischen den beiden Packages steht das Food, da es sowohl vom Profile als auch von der Ingredient verwendet wird und daher keinem der Packages eindeutig zugeordnet werden kann. Auch bei Food handelt es sich um ein Value Object, da es ebenfalls nur durch seine Werte definiert wird: Zwei Foods mit gleichem Namen sind gleich, daher besitzt es keine eigene Identität und da sich der Name auch nicht ändern wird, besitzt es auch keinen erkennbaren Lebenszyklus. 

\section{Aggregates}
Da alle Klassen im Package \href{https://github.com/anditru/quickie/tree/bb41442c7f1ffbfcd3117cd86a40f7932e543a33/3-quickie-domain/src/main/java/org/pinkcrazyunicorn/quickie/domain/recipe}{\code{recipe}} Eigenschaften eines Recipe abbilden, werden diese, also die Klassen Recipe, Ingredient, Unit, Quantity und Unit, zu einem Aggregate zusammengefasst. Die Root Entity ist dabei das Recipe selbst.

Ähnlich verhält es sich im Package \href{https://github.com/anditru/quickie/tree/bb41442c7f1ffbfcd3117cd86a40f7932e543a33/3-quickie-domain/src/main/java/org/pinkcrazyunicorn/quickie/domain/profile}{\code{profile}}: Auch hier bilden alle Klassen direkt oder indirekt Eigenschaften des Profiles ab, daher werden auch diese, also die Recipe, Food, OpinionAbout und Opinion bzw. deren Unterklassen zu einem Aggregate zusammengefasst. Die Root Entity ist hier das Profile.

Die Klasse Food nimmt eine Sonderstellung ein: Da sie von beiden Aggregates verwendet wird und daher zuvor auch schon keinem Package eindeutig zugeordnet werden konnte, ist sie Teil beider Aggregates.

\section{Repositories}
Für den Zugriff auf den persistenten Speicher werden zwei Repositories gemäß dem Grundsatz \enquote{Ein Repository pro Aggregate} definiert: Das ProfileRepository erlaubt den Zugriff auf das Profile, also die Root Entity des entsprechenden Aggregates. Analog hierzu ermöglicht das RecipeRepository Zugriff auf das Recipe als Root Entity des zugehörigen Aggregates. 

\chapter{Clean Architecture}
In diesem Kapitel wird die Architektur der entwickelten Software beschrieben. Diese wurde nach den in der Vorlesung besprochenen Prinzipien der Clean Architecture aufgebaut. Jeder der nachfolgenden Abschnitte behandelt eine Schicht. Die Schichten 1, 2 und 3 sind jeweils als ein eigenes Java-Module implementiert. Lediglich Schicht 0, welche die Plugins und das Main-Module mit der Main-Klasse der Applikation enthält, bildet eine Ausnahme: Hier entspricht jedem Plugin sowie dem Main-Module ein separates Java-Module, da diese komplett und leicht austauschbar sein sollen.

Die Clean Architecture hält neben den vier für dieses Projekt verwendeten Schichten noch eine fünfte bereit, den Abstraction Code. Auf diese Schicht, wurde für dieses Projekt jedoch verzichtet, da für die in der Domäne behandelten Themengebiete \enquote{Kochrezepte} und \enquote{Lebensmittel} kein domänenübergreifendes Wissen notwendig war, welches Teil dieser Schicht hätte sein müssen. 

\section{Schicht 3: Domain}
Diese Schicht befindet sich im Module \code{3-quickie-domain} und enthält die in \autoref{fig:class-diag-domain} dargestellten Klassen und Interfaces. Die enthaltenen Klassen implementieren die Entities und Value Objects der Domäne der Software, welche die Enterprise Business Logik der Software abbilden und damit die typischen Elemente des Domain Codes darstellen. Die enthaltenen Interfaces geben die notwendigen Methoden für die zugehörigen Repositories vor, welche gemäß der in der Vorlesung besprochenen Clean Architecture ebenfalls Teil des Domain Codes sind.

Der Code dieser Schicht bedient sich lediglich des Java-Standards und ist damit als zentrale und langlebigste Schicht der Software frei von jeglichen Abhängigkeiten.

\section{Schicht 2: Application}
Diese Schicht befindet sich im Module \code{2-quickie-application} und implementiert die eingangs beschriebenen Anwendungsfälle. Diese sind in den drei Services \code{ProfileService}, \code{MatchingService} und \code{RecipeService} gruppiert. Diese sind in \autoref{fig:class-diag-application} als Klassendiagramme dargestellt.

\begin{figure}[ht!]
    \includegraphics[width=0.98\columnwidth]{../diagrams/application_uml.pdf}
    \caption{Klassendiagramm der Applikationsschicht}
    \label{fig:class-diag-application}
\end{figure}

Die einzige Abhängigkeit des Modules besteht dabei auf den Domaincode der Software.

\section{Schicht 1: Adapters}
Die Adapterschicht wurde im Module \code{1-quickie-adapters} implementiert und erfüllt in der zugrundeliegenden Software zwei Aufgaben: Zum einen werden die Daten durch Mapper-Klassen von dem Format, welches die Services aus der Applikationsschicht liefern, in das Format übersetzt, welches die Plugins benötigen und umgekehrt. Zum anderen regelt sie die Kommunikation zwischen den Plugins und der Applikationsschicht. Um hierbei eine möglichst große Entkopplung zwischen den Plugins und der Applikationsschicht zu erreichen, geschieht diese Kommunikation eventbasiert.

\begin{figure}[ht!]
    \includegraphics[width=0.98\columnwidth]{../diagrams/adapter_uml.pdf}
    \caption{Vereinfachter Ausschnitt des Klassendiagramms der Adapterschicht}
    \label{fig:class-diag-adapter}
\end{figure}

\autoref{fig:class-diag-adapter} zeigt einen Ausschnitt des Klassendiagramms der Adapterschicht in vereinfachter Form. Die Adapterschicht besitzt vier Packages: \code{event}, \code{profile}, \code{recipe} und \code{persistence}. Letzteres wurde hier im Sinne der Übersichtlichkeit zunächst nicht dargestellt. 

Das Package \code{event} enthält Pure Fabrication Code, welcher ausschließlich die eventbasierte Kommunikation betrifft. Die Klasse \code{Event} repräsentiert hierbei ein Event, welches von der Benutzeroberfläche gesendet wird und besitzt einen \code{EventType}. Die Klasse \code{EventAnswer} bildet das Resultat der Verarbeitung eines Events ab, welches an die Benutzeroberfläche zurückgeschickt wird. Die Daten, welche mittels der \code{EventAnswer} an die Benutzeroberfläche geschickt werden, werden in einem separaten Objekt gehalten. Hierbei handelt es sich je nach Form der Daten um eine Instanz einer Klasse, welche das Interface \code{EventAnswerData} implementiert. Die Logik, welche beim Auftreten eines bestimmten Events aufgerufen wird, befindet sich in Callback-Klassen. Zu jedem \code{EventType} existiert eine entsprechende Callback-Klasse, welche das Interface \code{EventCallback} implementiert.

In den Packages \code{profile} und \code{recipe} werden zum einen die entsprechenden Callback-Klassen implementiert und zum anderen Mapper-Klassen, welche die von der Applikationsschicht erhaltenen Daten in das von der Benutzeroberfläche benötigte Format überführen.

Das zentrale Element der Adapterschicht ist der Controller. Dieser registriert beim Start der Applikation zunächst für jeden \code{EventType} die zugehörige Callback-Klasse bei der Benutzeroberfläche, welche das Interface \code{UI} implementieren muss. Er bildet außerdem den Einstiegspunkt bei der Verarbeitung eines Events von der Benutzeroberfläche, indem er das aufgetretene Event von der Benutzeroberfläche abfragt und dessen Verarbeitung durch den Aufruf der Methode \code{handleEvent} in Gang setzt. 

\begin{figure}[ht!]
    \includegraphics[width=0.98\columnwidth]{../diagrams/adapter_sequence.pdf}
    \caption{Verarbeitung eines User-Events}
    \label{fig:squence-diag-adapter}
\end{figure}

\autoref{fig:squence-diag-adapter} zeigt beispielhaft den Ablauf bei der Verarbeitung eines Events vom Typ \code{viewRecipes}. Die beteiligten Objekte sind hier farblich nach ihrer Schichtzugehörigkeit markiert: Grün steht für die Adapterschicht, Blau für die Applikationsschicht und Rot für ein Plugin.

Das Package \code{persistence} bildet die Schnittstelle zwischen der Applikationsschicht und Plugins für die Persistenz. Hier werden Interfaces definiert, welche die zu persistierenden Klassen implementieren müssen, nämlich \code{PersistentProfile}, \code{PersistentRecipe} und \code{PersistentIngredient}. Außerdem werden die Mapper-Klassen \code{PersistentProfileMapper} und \code{PersistenRecipeMapper} implementiert, welche die Domänenobjekte in ihre persistenten Gegenstücke transformieren und umgekehrt. Ferner werden in dem Package die Interfaces \code{PersistentProfileFactory} und \code{PersistentRecipeFactory} definiert, welche dann im Persistenz-Plugin implementiert werden und in den Mapper-Klassen verwendet werden, um leere \code{PersistentProfile}s und \code{PersistentRecipe}s zu erzeugen. Zuletzt werden hier auch die beiden im Domaincode definierten Interfaces \code{RecipeRepository} und \code{ProfileRepository} in den beiden abstrakten Klassen \code{PersitentRecipeRepository} und \code{PersistentProfileRepository} implementiert, von welchen dann die Repositories in einem Persistenz-Plugin erben.

\section{Schicht 0: Plugins}

\subsection{Plugin \acs{CLI}}
Dieses Plugin, welches im Module \code{0-quickie-plugin-cli} implementiert wurde und die Benutzeroberfläche der Software bildet, enthält lediglich die Klasse \code{CommandLineUI}. Diese implementiert das Interface \code{UI} aus der Adapterschicht. Abgesehen von dieser besitzt dieses Plugin nur eine weitere Abhängigkeit: Die Library Apache Commons CLI, welche zur leichteren Implementierung der \ac{CLI} verwendet wird. Das Plugin kommuniziert lediglich über die Eventfunktionalität der Adapterschicht mit dem Kern der Software, ist damit nur sehr lose an den selben gekoppelt und kann leicht ausgetauscht werden.

\subsection{Plugin Gson}

\subsection{Plugin JPA}
Dieses Plugin wurde im Module \code{0-quickie-plugin-jpa} implementiert und realisiert die Persistenz der Software mit Hilfe der \ac{JPA} und deren Implementierung Hibernate. Auch dieses Plugin besitzt nur eine Abhängigkeit auf die Adapterschicht, da hier die dort im Package \code{persistence} definierten Interfaces für Factories und die zu persistierenden Objekte Recipe, Ingredient und Profile implementiert werden. Letztere enthalten außerdem die \ac{JPA}-Annotationen. Ferner enthält das Plugin die Klassen \code{JPARecipeRepository} und \code{JPAProfileRepository}, welche von den beiden abstrakten Repository-Klassen aus der Adapterschicht erben. Schließlich enthält das Plugin noch den \code{PersistenceMananger}, einen Singleton, welcher den \code{EntityManager} von \code{JPA} für die Repositories liefert.

\subsection{Plugin Scraper}
In diesem Plugin wird die Extraktion der Rezepte aus den entsprechenden Webseiten im Module \code{0-quickie-plugin-scraper} implementiert. Hierzu werden die Webseiten zunächst durch den \code{CachedDownloader} heruntergeladen. Für das Parsen der Webseiten wird der \code{HensslerScraper} verwendet, welcher die Rezeptdaten aus dem HTML-Text extrahiert. Diese werden dann durch die \code{HensslerDatasource}, welche das in der Applikationsschicht definierte Interface \code{Datasource} implementiert, für die Applikationsschicht zugänglich gemacht. Die Daten durchlaufen damit nicht die Adapterschicht, da der Scraper die Daten problemlos direkt in dem von der Applikationsschicht benötigten Format liefern kann.
% Redet mit Adapterschicht
% Hat ganz schön viel Logik 

\subsection{Plugin Main}
Das Module \code{0-quickie-main} enthält lediglich die Main-Klasse der Applikation, welche alle nötigen Klassen instanziiert und ggf. injected. Zuletzt wird hier auch die Methode \code{run} des Controllers aus der Adapterschicht ausgeführt und somit die Verarbeitung von Events gestartet. Dieses Module besitzt folglich Abhängigkeiten auf alle anderen Modules der Software und befindet sich daher in der äußersten Schicht.
% Begründung dass das ein Plugin ist

\chapter{Unit Tests}
Zur frühzeitigen Erkennung von Fehlern und der Sicherung der Qualität der Software wurden für das Projekt einige Unit Tests implementiert. Diese beschränken sich im Wesentlichen jedoch auf die Applikationssicht der Software, da dort der größte Teil der tatsächlichen Businesslogik implementiert ist. 

Für die Peripherie der Software wurde zunächst auf Unit Tests verzichtet, da dieser Teil des Programms ohnehin leicht und häufig austauschbar sein soll und daher von geringerer Wichtigkeit ist, sodass ihm auch beim Testen eine geringere Priorität beigemessen wurde. Lediglich in der Adapterschicht wurden für eine Klasse, den \code{OpinionMapper}, als einfaches Beispiel zu Übungszwecken einige Unit Tests implementiert. Auch für den Domaincode wurde zunächst auf Unit Tests verzichtet, da sich die dort implementierte Businesslogik hauptsächlich auf die Definition von Klassen und ihre Attribute beschränkt, es existieren jedoch keine Methoden, welche tatsächlich komplexere Logik enthalten.

\section{ATRIP-Regeln}
Bei den ATRIP-Regeln handelt es sich um fünf grundlegende Regeln für die Erstellung guter Unit Tests. Inwiefern diese für das vorliegende Projekt eingehalten wurden, wird in den nachfolgen Abschnitten aufgezeigt.

\paragraph{Automatic} Alle implementierten Unit Tests laufen vollständig eigenständig ab, da keinerlei manuelle Eingriffe, etwa in Form von Werteeingaben, notwendig sind. Außerdem überprüfen alle Tests ihre Ergebnisse durch Assertions automatisch.

\paragraph{Thorough} Da es sich bei dieser Regel um eine \enquote{weiche} Regel handelt, ist eine eindeutige Entscheidung, ob alles Notwendige getestet wurde und damit die Regel erfüllt ist schwierig. Als \enquote{notwendig} wurde für das vorliegende Projekt die wesentliche Businesslogik und damit die Applikationsschicht definiert. Diese wird mit einer hohen Code-Coverage getestet, sodass die implementierten Tests durchaus als gründlich bezeichnet werden können (vgl. \autoref{sec:code-coverage}). Allerdings wurden keine Tests für spezifische Softwarefehler implementiert, da sich die Software zu dem Zeitpunkt, zu dem dieses Dokument verfasst wird, noch nicht ausgiebig von Endanwendern getestet wurde und daher bisher keine Softwarefehler gemeldet wurden.

\paragraph{Repeatable} Die implementierten Unit Tests sind jederzeit automatisch durchführbar und liefern das gleiche Ergebnis, da sie weder zeit- noch zufallsabhängige Komponenten beinhalten und auch keine Abhängigkeiten auf Dateisysteme, Datenbanken oder ähnliches besitzen bzw. diese durch Mocks ersetzt werden, welche stets die gleichen Daten liefern (vgl. \autoref{sec:mocks}).

\paragraph{Independent} Die implementierten Unit Tests sind jederzeit in beliebiger Reihenfolge ausführbar. Dies wird sichergestellt, indem alle Tests strikt nach der AAA-Normalform aufgebaut sind, also jeder Test in seiner ersten Phase seine eigene \enquote{Testwelt} initialisiert. Im Produktivbetrieb gibt es Abhängigkeiten auf persistierte Daten. Um dieses in den Unit Tests zu verhindern, werden jegliche Persistenzzugriffe auf Mocks ausgeführt, welche in der ersten Phase jedes Tests neu trainiert werden.

\paragraph{Professional} Um dieses Kriterium zu erfüllen, wurde darauf geachtet, dass der Testcode möglichst leicht verständlich ist. Hierzu wurden beispielsweise die Namen der einzelnen Tests nach einem bestimmten immer gleichen Muster gewählt: Jede Testmethode beginnt mit dem Wort \enquote{should} und beschreibt das erwartete Verhalten des zu testenden Codes, zum Beispiel \code{shouldFindExpectedMatchingRecipes}. Außerdem wurde das jeweils zu testende Objekt stets mit \code{codeUnderTest} bezeichnet. Ferner folgen alle Tests wie zuvor erwähnt der AAA-Normalform, bestehen also stets aus den gleichen drei Phasen \enquote{Arrange}, \enquote{Act} und \enquote{Assert} besteht. Für die letzte Phase wurde außerdem die Library AssertJ verwendet, welche eine fluent \ac{API} für besser lesbare Assertions bietet.

\section{Code-Coverage}
\label{sec:code-coverage}
Wie eingangs erwähnt beschränken sich die implementierten Unit Tests im Wesentlichen auf die Applikationsschicht, da dort der Großteil der Businesslogik vorhanden ist. Hier wird jedoch nach beiden gängigen Standards eine sehr hohe Code-Coverage erreicht. \autoref{tab:code-coverage} zeigt eine Messung der Code-Coverage der \acs{IDE} Intellij IDEA für das Package \code{org.pinkcrazyunicorn.quickie.application}, welches den gesamten Code der Applikationsschicht enthält. So beträgt die Line-Coverage hier 98\% und die aussagekräftigere Branch-Coverage 80\%.

\begin{table}[ht]
    \begin{tabular}{|l|c|c|}
        \hline
        \multicolumn{1}{|c} {Package / Klasse} & \multicolumn{1}{|c|} {Line-Coverage} & \multicolumn{1}{c|}{Branch-Coverage}\\
        \hline
        \begin{minipage}{8cm}\dirtree{%
            .1 org.pinkcrazyunicorn.quickie.application.
            .2 org.pinkcrazyunicorn.quickie.application.profile.
            .3 ProfileService.
            .2 org.pinkcrazyunicorn.quickie.application.recipe.
            .3 MatchingService.
            .3 RecipeService.
        }\end{minipage}
        &
        \DTsetlength{0pt}{0pt}{0pt}{0pt}{0pt}
        \begin{minipage}{3cm}\centering\dirtree{%
            .1 98\%.
            .2 100\%.
            .3 100\%.
            .2 98\%.
            .3 98\%.
            .3 100\%.
        }\end{minipage}
        &
        \DTsetlength{0pt}{0pt}{0pt}{0pt}{0pt}
        \begin{minipage}{3cm}\centering\dirtree{%
            .1 80\%.
            .2 100\%.
            .3 100\%.
            .2 80\%.
            .3 77\%.
            .3 100\%.
        }\end{minipage} \\
        \hline
    \end{tabular}
    \caption{Code Coverage der Applikationsschicht}
    \label{tab:code-coverage}
\end{table}

\section{Einsatz von Mocks}
\label{sec:mocks}
Mocks spielen eine zentrale Rolle bei der Implementierung von Unit Tests, da sie es ermöglichen, eine Klasse isoliert zu testen. Sie ersetzen dabei die Abhängigkeiten einer Klasse als Objekte mit der für den jeweiligen Test minimal notwendigen Funktionalität. Auch für die im Rahmen dieses Projekts in der Applikationsschicht implementierten Unit Tests kommen Mocks zum Einsatz, welche mit Hilfe des Mocking-Frameworks EasyMock erzeugt werden.

Getestet werden in dieser Schicht der \code{ProfileService} zur Verwaltung der Benutzerprofile, der \code{RecipeService} zur Verwaltung der Rezepte und der \code{MatchingSerivce}, welcher Rezepte mit hoher Übereinstimmung in den Zutaten bestimmt. Alle drei Services besitzen Abhängigkeiten auf mindestens ein Repository für den Zugriff auf Entities und Value Objects. Diese werden für die Unit Tests durch Mocks ersetzt, welche die für den jeweiligen Test notwendigen Daten liefern.

\chapter{Entwurfsmuster}
Entwurfsmuster dienen in der Softwareentwicklung zur Lösung wiederkehrender Probleme. Es handelt sich hierbei jedoch nicht um fertigen Code oder feste Designs, sondern lediglich um Lösungsansätze und Schablonen für typische Probleme in der Softwareentwicklung. Auch für dieses Projekt wurden Entwurfsmuster eingesetzt. Eines dieser Entwurfsmuster soll im nachfolgenden Abschnitt genauer erläutert werden.

\section{Fabrikmethode}
Die Fabrikmethode gehört zur Kategorie der Erzeugungsmuster. Diese beziehen sich stets auf die Erzeugung von Instanzen und kommen immer dann zum Einsatz, wenn die direkte Objekterstellung durch den Konstruktor ungewollte Komplexität mit sich bringen würde. 
Bei der Verwendung einer Fabrikmethode wird ein Objekt nicht durch den Konstruktor erzeugt, sondern durch den Aufruf einer speziellen Methode, die, wie das Entwurfsmuster selbst, als Fabrikmethode bezeichnet wird.

Für das vorliegende Projekt, werden Fabrikmethoden zur Erzeugung persistenter Objekte verwendet. Die Notwendigkeit hierfür ergibt sich daraus, dass die Klassen, deren Instanzen persistiert werden können, im \acs{JPA}-Plugin implementiert werden. Gleichzeitig sollen die Objekte dieser Klassen jedoch in der Adapterschicht erzeugt werden, da diese für das Mapping zwischen verschiedenen Datenformaten verantwortlich ist. Folglich wird in der Adapterschicht eine Methode zur Erzeugung der persistenten Objekte benötigt. Um dabei Abhängigkeiten der Adapterschicht auf das \acs{JPA}-Plugin zu vermeiden, werden nach dem Prinzip der Dependency-Inversion die Interfaces \code{PersistentProfileFactory} und \code{PersistentRecipeFactory} in der Adapterschicht definiert. Da in einem Interface jedoch keine Konstruktoren definiert werden können, werden dort statt dessen die Fabrikmethoden definiert. Diese werden dann im \acs{JPA}-Plugin in den Klassen \code{JPAProfileFactory} und \code{JPARecipeFactory} implementiert und dann in den Mapperklassen der Adapterschicht sowie den Implementierungen der Repositories im \acs{JPA}-Plugin verwendet. Das gesamte Konzept wird in \autoref{fig:class-diag-factory} als Klassendiagramm visualisiert.

\begin{figure}[ht!]
    \includegraphics[width=0.98\columnwidth]{../diagrams/factory_uml.pdf}
    \caption{Klassendiagramm zur Fabrikmethode}
    \label{fig:class-diag-factory}
\end{figure}

\autoref{fig:squence-diag-factory} zeigt anhand eines Sequenzdiagramms den Ablauf der Erzeugung und Persistierung eines Profiles mit Hilfe der Fabrikmethode \code{createEmpty()}. Auch hier sind die beteiligten Objekte farblich nach ihrer Schichtzugehörigkeit markiert: Grün steht für die Adapterschicht, Blau für die Applikationsschicht und Rot für ein Plugin.

\begin{figure}[ht!]
    \includegraphics[width=0.98\columnwidth]{../diagrams/factory_sequence.pdf}
    \caption{Sequenzdiagramm zur Fabrikmethode}
    \label{fig:squence-diag-factory}
\end{figure}

\chapter{Programming Principles}

\section{SOLID}

\section{GRASP}

\section{DRY}
DRY steht für \enquote{Don't Repeat Yourself} und ist der Titel eines Programmierprinzips, welches besagt, dass redundanter Code vermieden werden sollte. Grund dafür ist, dass dadurch Wissen zentral festgehalten wird. Sollte das festgehaltene Wissen verändert werden müssen, so muss dieses folgend auch nur an einer Stelle geändert werden. Im Folgenden sollen Code-Beispiele diskutiert werden, in welchen das DRY-Prinzip eingehalten oder auch nicht eingehalten wurde.

Ein Beispiel für das Einhalten des DRY-Prinzips ist in der Verifizierung der Parameter eines Callbacks zu sehen. Hier wird zentral in jedem Callback an einer Stelle definiert, welche Parameter benötigt werden und das CLI Plugin nutzt diese Informationen, um die übergebenen Parameter auf Vollständigkeit für den entsprechenden Callback zu prüfen sowie um die Benutzerhilfe auszugeben.

Ein Negativbeispiel für das Nichteinhalten des DRY-Prinzips ist in der Implementierung der zu persistierenden Entitäten zu sehen. Diese werden einmal zentral in der Domäne definiert, während es dann jedoch in der Adapterschicht noch die Interfaces \code{PersistentProfile} und \code{PersistentRecipe} gibt, welche durch Getter- und Setter-Methoden die notwendigen und zu persistierenden Eigenschaften der Entitäten definieren. Diese werden abschließend im \code{persistence} Plugin implementiert. Dadurch wird die Struktur einer Entität einmal in der Domäne und einmal in der Persistenz definiert, es gibt eine Dopplung der Informationen, welche Attribute zu einer Entität gehören. Diese Verletzung des DRY-Prinzips wurde begangen, um die Persistenz-Darstellung von der Arbeitsdarstellung zu entkoppeln, wodurch die Persistenz deutlich flexibler gestaltet werden kann.

Ein weiteres simpleres Negativbeispiel findet sich in Methoden des \code{JPAProfileRepository} und \code{JPARecipeRepository}. In diesen wird, falls ein \code{PersistentReicipe} oder \code{PersistentProfile} übergeben wird, immer erstmal geprüft, ob dieses eine Instanz eines \code{JPARecipe} beziehungsweise \code{JPAProfile} ist. Diese Überprüfung hätte in eine eigene Methode ausgelagert werden, um Flüchtigkeitsfehler durch die Redundanz des Codes zu verhindern und die Wartung zu erleichtern.

Ein Beispiel für eine solche ausgelagerte Methode findet sich in der Klasse \code{CommandLineUI} mit der Methode \code{printHelp}. Diese Methode lagert zentral aus, wie die Hilfe ausgegeben werden soll und erfüllt somit das DRY-Prinzip.

% ---- Literaturverzeichnis
\cleardoublepage
\renewcommand*{\chapterpagestyle}{plain}
\pagestyle{plain}
\pagenumbering{Roman}                   % Römische Seitenzahlen
\setcounter{page}{\numexpr\value{savepage}+1}
\printbibliography[title=Literaturverzeichnis]

% ---- Anhang
\appendix
%\clearpage
%\pagenumbering{Roman}  % römische Seitenzahlen für Anhang

\newpage
\end{document}
